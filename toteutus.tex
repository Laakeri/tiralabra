\documentclass[a4paper, 11pt]{article}
\usepackage{comment} % enables the use of multi-line comments (\ifx \fi) 
\usepackage{lipsum} %This package just generates Lorem Ipsum filler text. 
\usepackage{fullpage} % changes the margin
\usepackage[utf8]{inputenc}
\usepackage{amsmath}
\usepackage{amsfonts}
\usepackage{hyperref}
\title{Mincost-flow-algoritmit toteutusdokumentti}
\author{Tuukka Korhonen}
\date{\today}
\begin{document}
\maketitle
\noindent
Toteutettiin algoritmit \textsc{SAPSPFA}, \textsc{SAPDijkstra} ja \textsc{ScalingCirculation}.
Algoritmeista ja ongelman määrittelystä tietoa tiedostossa algoritmit.pdf ja käyttöohje
githubin markdown muodossa tiedostossa README.md. Testausdokumentti tiedostossa testaus.pdf.
Yksikkötestejä ei ole koska testasin algoritmeja ICPC 2015 tehtävän Catering testeillä
ja testasin miljoonilla satunnaisgeneroiduilla testeillä että algoritmit antavat
saman vastauksen.
Lisäksi olen käyttänyt suunnilleen samaa implementaatiota useammissakin kisakoodaustehtävissä
jotka on testattu hyvällä testidatalla joten olen varma että kyseinen implementaatio
on kunnossa. Kisakoodausversio implementaatiosta löytyy \hyperref[https://github.com/Laakeri/contestlib/blob/master/src/graph/mincostflow.cpp]{https://github.com/Laakeri/contestlib/blob/master/src/graph/mincostflow.cpp}
\\\\
\noindent
Työssä piti toteuttaa itse dynaaminen taulukko tietorakenne ja jonkinlainen heap
tietorakenne Dijkstraa varten. Käytin segmenttipuuhun perustuvaa staattista heappia
joka toimii kivasti Dijkstran kanssa ja on helppo koodata ja nopea koska muistia ei varata dynaamisesti. 
Staattisessa 
heapissa on siis $n$ paikkaa, ja voidaan $O(\log n)$ ajassa asettaa jonkun paikan arvo
ja $O(1)$ ajassa katsoa pienin arvo ja millä paikalla se on. Muistia käytetään $O(n)$.
\end{document}




