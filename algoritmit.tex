\documentclass[a4paper, 11pt]{article}
\usepackage{comment} % enables the use of multi-line comments (\ifx \fi) 
\usepackage{lipsum} %This package just generates Lorem Ipsum filler text. 
\usepackage{fullpage} % changes the margin
\usepackage[utf8]{inputenc}
\usepackage{amsmath}
\usepackage{amsfonts}
\title{Mincost-flow-algoritmit}
\author{Tuukka Korhonen}
\date{\today}
\begin{document}
\maketitle
\noindent
\section*{Mincost-flow}
Annetaan virtausverkko $G = (V, E, U, C, s, t)$, jossa $U(e) \in \mathbb{Z_+}$ on 
kaaren $e$ kapasiteetti, $C(e) \in \mathbb{R}$ on kaaren $e$ hinta ja $s$ ja $t$
ovat lähde- ja viemärisolmut. Annetaan lisäksi kokonaisluku $k$ joka tarkoittaa
kuinka paljon virtausta lähetetään. Merkinnällä $uv$ tarkoitetaan kaarta solmusta
$u$ solmuun $v$. Oletetaan tämä merkintä yksikäsitteiseksi eli 
kahden solmun välillä ei
saa olla useaa kaarta. Lisäksi oletetaan että jos verkossa on kaari $uv$, siinä
ei ole kaarta $vu$. Nämä oletukset ovat vain teoriaa varten ja niitä ei tehdä
implementaatiossa. \textsc{Mincost-flow}-ongelmassa tavoitteena on löytää virtaus jota 
nyt merkitään funktiona
$f : E \rightarrow \mathbb{Z}$ joka: \\\\ Minimoi $$\sum_{uv \in E} C(uv) f(uv)$$\\
Toteuttaa ehdot:\\ Säilyvyysehto
$$\sum_{v \in V} f(uv) = \sum_{v \in V} f(vu) \text{ kaikilla } u \in V \setminus \{s, t\}$$
\\Kapasiteettiehto
$$0 \le f(uv) \le U(uv) \text{ kaikilla } uv \in E$$
\\Virtauksen määrä
$$\sum_{v \in V} f(sv) = k \text{ ja } \sum_{v \in V} f(vt) = k$$
Lisäksi määritellään että \textsc{Mincost-flow}-ongelmassa verkossa ei saa olla
suunnattuja syklejä, joiden hintojen summa on negatiivinen.\\\\

\noindent
Toisin sanoen lähetetään $k$ yksikköä virtausta lähteestä $s$ viemäriin $t$ niin
että virtauksen käyttämien kaarien hinta on mahdollisimman pieni. Tämän voi ajatella
esimerkiksi ongelmana jossa verkon solmut ovat kaupunkeja. Kaupungissa $s$ on tehdas
josta pitää lähettää joka päivä $k$ tonnia tavaraa varastoon joka on kaupungissa $t$.
Lisäksi tiedetään että kaupunkien välillä on junayhteyksiä (eli kaaria). Kaaren $uv$
kapasiteetti kertoo paljonko sillä junayhteydellä voi kuljettaa tavaraa päivässä
kaupungista $u$ kaupunkiin $v$. Kaarien hinnat kertovat paljonko maksaa yhden tonnin
tavaraa kuljettaminen sillä junayhteydellä. Nyt halutaan kuljettaa $k$ tonnia tavaraa
tehtaasta varastoon päivittäin ja minimoida  kuljetuksen hinta.\\\\

\noindent
Syy negatiivisten syklien kieltämiseen on, että niin saadaan yksinkertaisempia toimivia
algoritmeja ja tämä määritelmä sopii ongelman luonteeseen
jossa lähetetään tavaraa lähteestä viemäriin. Jos negatiivisia syklejä saisi
olla, niin optimaalinen ratkaisu voisi pyörittää virtausta jossain muualla kuin reiteillä
lähteestä viemäriin. Myöhemmin esitetään \textsc{Mincost Circulation} ongelma, jossa
negatiiviset syklit ovat sallittuja.\\

\section*{Shortest-Augmenting-Path-algoritmi}
Shortest-Augmenting-Path-algoritmi (lyhyemmin SAP-algoritmi) etsii
mincost-flown lähettämällä
virtausta lyhyintä reittiä kaarien hintojen suhteen lähteestä viemäriin.
Tällaista lyhyintä reittiä kutsutaan \textit{augmentoivaksi poluksi}. Algoritmi 
on muunnelma Ford-Fulkerson-algoritmista 
maksimivirtauksen etsimiseen. Intuitiota saa miettimällä miten mincost-flow etsittäisiin
jos virtausta lähetettäisiin vain yksi yksikkö: se olisi lyhyin reitti lähdesolmusta
viemärisolmuun.\\\\
\noindent
\subsection*{Jälkiverkko}
Sanotaan että $G_f$ on virtauksen $f$ jälkiverkko. Määritellään kapasiteetit
jälkiverkossa:\\
$$U_f(uv) = \begin{cases} U(uv) - f(uv) \text{ jos } uv \in E\\
f(vu) \text{ jos } vu \in E\\
0 \text{ muuten}
\end{cases}$$
Määritellään hinnat jälkiverkossa:\\
$$C_f(uv) = \begin{cases} C(uv) \text{ jos } uv \in E\\
-C(vu) \text{ jos } vu \in E\\
0 \text{ muuten}
\end{cases}$$
Sanotaan että kaari $uv$ on jälkiverkossa jos $U_f(uv) > 0$.\\\\
\noindent
Nyt jos jälkiverkossa on polku $s \rightsquigarrow t$ jonka pienin kapasiteetti 
on $u$, niin sitä polkua pitkin voidaan lähettää maksimissaan $u$ yksikköä
virtausta ja yhden lähetetyn virtausyksikön hinnaksi tulee polun kaarien hintojen summa.
Tämä kasvattaa virtauksen määrää $s$:stä $t$:hen. Toinen sallittu tapa muuttaa virtausta
on lähettää virtausta jotain jälkiverkon sykliä pitkin, maksimissaam syklin pienimmän
kapasiteetin verran. Laskemalla voidaan tarkistaa että nämä muutokset säilyttävät
virtauksen säilyvyys- ja kapasiteettiehdon.\\\\
\noindent
\subsection*{Virtauksen optimaalisuus}
\textbf{Lemma 1:} $k$-virtaus on optimaalinen jos jälkiverkossa ei ole sykliä jonka
hinta on negatiivinen.\\\\
\noindent
\textbf{Todistus 1:} Olkoon $f$ joku $k$-virtaus niin että $G_f$:ssä ei ole negatiivista
sykliä. Olkoon $f'$ joku $k$-virtaus jonka hinta on pienempi kuin $f$:n hinta. Määritellään
$(f' - f)(uv) = f'(uv) - f(uv)$. Nyt $f' - f$ on virtaus joka toteuttaa säilyvyysehdon kaikissa 
solmuissa, mukaan lukien lähde- ja viemärisolmut, koska $f'$ ja $f$ ovat molemmat $k$-virtauksia.
Virtauksen $f' - f$ hinta on negatiivinen. Lisäksi $f' - f$ on sallittu
virtaus $G_f$:ssä. Säilyvyysehdon toteuttavan virtauksen voi hajottaa joukoksi syklejä
jotka toteuttavat säilyvyysehdon koska jos otetaan mikä tahansa
sykli virtauksesta pois, niin jäljelle jää virtaus joka toteuttaa säilyvyysehdon.
Siispä kun $f' - f$ hajotetaan sykleiksi, niin ainakin yhden syklin hinta on negatiivinen.
Sitä sykliä pitkin voidaan lähettää virtausta $G_f$:ssä, joten $G_f$:ssä on negatiivinen sykli.\\\\
\noindent
\textbf{Lemma 2:} Jos jälkiverkossa ei ole negatiivista sykliä, niin lähettämällä virtausta
halvinta $s \rightsquigarrow t$ polkua pitkin sinne ei synny negatiivista sykliä.\\\\
\noindent
\textbf{Todistus 2:} Sanotaan että solmujen potentiaalifunktio on funktio 
$p: V \rightarrow \mathbb{R}$, joka toteuttaa
$p(u) + C_f(uv) \ge p(v)$ kaikilla jälkiverkon kaarilla $uv$. Jos $G_f$:ssä ei ole 
negatiivista sykliä niin tällainen 
funktio on olemassa, sillä esim. lyhyin reitti jostain solmusta kelpaa täksi funktioksi.
Nyt määritellään kaaren $uv$ vähennetty hinta $C_{fr}(uv) = p(u) + C_f(uv) - p(v)$.
Nähdään että syklin hinta vähennetyissä hinnoissa on sama kuin syklin hinta. Nähdään myös
että kaikki vähennetyt hinnat ovat epänegatiivisia, josta on helppo nähdä että jos on
olemassa validi potentiaalifunktio, niin negatiivisia syklejä ei ole.\\
Kun lähetetään virtausta lyhyintä $s \rightsquigarrow t$ polkua pitkin, voidaan
potentiaalifunktiota päivittää seuraavalla tavalla: Olkoon $sp(u)$ lyhyimmän reitin pituus
solmusta $s$ solmuun $u$ kun tarkastellaan reitin pituutta vähennetyissä hinnoissa.
Jos ei ole polkua solmusta $s$ solmuun $u$, niin $sp(u)$ on joku riittävän suuri vakio.
Uusi potentiaalifunktio on $p'(u) = p(u) + sp(u)$. Tämä toteuttaa potentiaalifunktion
ehdot kaikilla kaarilla jotka olivat jo verkossa koska $sp(u) + C_{fr}(uv) \ge sp(v)$
koska $sp$ on lyhyin reitti-funktio joten $sp(u) + p(u) + C_f(uv) - p(v) \ge sp(v)$
joten $sp(u) + p(u) + C_f(uv) \ge sp(v) + p(v)$ joten $p'(u) + C_f(uv) \ge p'(v)$.
Lisäksi verkkoon saattoi ilmestyä uusia kaaria kun lyhyintä $s \rightsquigarrow t$ polkua
päinvastaiseen suuntaan menevien kaarien kapasiteetti kasvoi. Kun $u$ ja $v$ ovat
lyhyimmällä polulla peräkkäin olevat solmut, niin $sp(u) + C_{fr}(uv) = sp(v)$ joten
$p'(u) + C_f(uv) = p'(v)$ joten $p'(v) - C_f(uv) = p'(u)$ eli potentiaalifunktio
toimii kaikilla uusilla kaarilla jotka ilmestyivät jälkiverkkoon.\\\\
SAP-algoritmi siis lähettää virtausta jälkiverkon lyhintä $s \rightsquigarrow t$ polkua
pitkin niin kauan
kunnes löytää $k$-virtauksen tai maksimivirtauksen. Yllä todistettiin että algoritmi
pitää joka vaiheessa yllä optimaalista virtausta.\\\\
\subsection*{Aikavaativuus}
Tässä määritelty SAP-algoritmi ei anna mitään tietoa tarvittavien augmentoivien
polkujen määrästä. Nähdään kuitenkin että virtauksen määrä kasvaa ainakin yhdellä, joten
augmentoivia polkuja on maksimissaan $k$. Periaatteessa virtausta voi lähettää 
korkeintaan $Um$ yksikköä, jossa $U$ on verkon suurin kapasiteetti. Lyhyimmän reitin
etsiminen ei ole ihan suoraviivaista koska verkossa on negatiivisa kaaria. Lyhyimmän reitin
verkossa jossa on negatiivisia kaaria
voi etsiä Bellman-Ford-algoritmilla, jonka aikavaativuus on $O(nm)$.
Kokonaisaikavaativuudeksi tulee $O(Um^2n)$. Käytännössä käytetään
SPFA-algoritmia, joka on käytännössä hyvin nopeasti toimiva versio Bellman-Fordista.
\\\\
\noindent
Todistuksessa 2 käytettiin solmujen potentiaalifunktiota. Nähtiin miten potentiaalifunktion
avulla saadaan muutettua kaikkien kaarien pituudet epänegatiivisiksi ja miten potentiaalifunktiota
voi päivittää helposti lyhyimpien reittien perusteella. Minkä tahansa $s \rightsquigarrow t$
polun pituus vähennetyissä hinnoissa on polun pituus $+ p(s) - p(t)$, eli lyhyimmän 
$s \rightsquigarrow t$ polun voi yhtä hyvin hakea vähennettyjen hintojen verkossa.
Saadaan algoritmi: etsitään ensin joku potentiaalifunktio Bellman-Ford-algoritmilla.
Sen jälkeen käytetään Dijkstraa vähennettyjen hintojen verkossa augmentoivien polkujen
löytämiseen ja päivitetään potentiaalifunktiota. Algoritmin aikavaativuss on $O(nm + Um^2 log n)$.
\end{document}




