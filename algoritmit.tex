\documentclass[a4paper, 11pt]{article}
\usepackage{comment} % enables the use of multi-line comments (\ifx \fi) 
\usepackage{lipsum} %This package just generates Lorem Ipsum filler text. 
\usepackage{fullpage} % changes the margin
\usepackage[utf8]{inputenc}
\usepackage{amsmath}
\usepackage{amsfonts}
\title{Mincost-flow-algoritmit}
\author{Tuukka Korhonen}
\date{\today}
\begin{document}
\maketitle
\noindent
\section*{Mincost-flow}
Annetaan virtausverkko $G = (V, E, U, C, s, t)$, jossa $U(e) \in \mathbb{Z_+}$ on 
kaaren $e$ kapasiteetti, $C(e) \in \mathbb{R}$ on kaaren $e$ hinta ja $s$ ja $t$
ovat lähtö- ja kohdesolmut. Annetaan lisäksi kokonaisluku $k$ joka tarkoittaa
kuinka paljon virtausta lähetetään. Merkinnällä $uv$ tarkoitetaan kaarta solmusta
$u$ solmuun $v$. Oletetaan tämä merkintä yksikäsitteiseksi eli 
kahden solmun välillä ei
saa olla useaa kaarta. Lisäksi oletetaan että jos verkossa on kaari $uv$, siinä
ei ole kaarta $vu$. Nämä oletukset ovat vain teoriaa varten ja niitä ei tehdä
implementaatiossa. \textsc{Mincost-flow}-ongelmassa tavoitteena on löytää minimihintainen virtaus jota 
nyt merkitään funktiona
$f : E \rightarrow \mathbb{Z}$ joka: \\\\ Minimoi $$\sum_{uv \in E} C(uv) f(uv)$$\\
Toteuttaa ehdot:\\ Säilyvyysehto
$$\sum_{v \in V} f(uv) = \sum_{v \in V} f(vu) \text{ kaikilla } u \in V \setminus \{s, t\}$$
\\Kapasiteettiehto
$$0 \le f(uv) \le U(uv) \text{ kaikilla } uv \in E$$
\\Virtauksen määrä
$$\sum_{v \in V} f(sv) = k \text{ ja } \sum_{v \in V} f(vt) = k$$
Lisäksi määritellään että \textsc{Mincost-flow}-ongelmassa verkossa ei saa olla
suunnattuja syklejä, joiden hintojen summa on negatiivinen.\\\\

\noindent
Toisin sanoen lähetetään $k$ yksikköä virtausta lähtösolmusta $s$ kohdesolmuun $t$ niin
että virtauksen käyttämien kaarien hinta on mahdollisimman pieni. Tämän voi ajatella
esimerkiksi ongelmana jossa verkon solmut ovat kaupunkeja. Kaupungissa $s$ on tehdas
josta pitää lähettää joka päivä $k$ tonnia tavaraa varastoon joka on kaupungissa $t$.
Lisäksi tiedetään että kaupunkien välillä on junayhteyksiä (eli kaaria). Kaaren $uv$
kapasiteetti kertoo paljonko sillä junayhteydellä voi kuljettaa tavaraa päivässä
kaupungista $u$ kaupunkiin $v$. Kaarien hinnat kertovat paljonko maksaa yhden tonnin
tavaraa kuljettaminen sillä junayhteydellä. Nyt halutaan kuljettaa $k$ tonnia tavaraa
tehtaasta varastoon päivittäin ja minimoida  kuljetuksen hinta.\\\\

\noindent
Syy negatiivisten syklien kieltämiseen on, että saadaan yksinkertaisempia toimivia
algoritmeja ja tämä määritelmä sopii ongelman luonteeseen
jossa lähetetään tavaraa lähtösolmusta kohdesolmuun. Jos negatiivisia syklejä saisi
olla, optimaalinen ratkaisu voisi pyörittää virtausta jossain muualla kuin reiteillä
lähtösolmusta kohdesolmuun. Myöhemmin esitetään \textsc{Mincost Circulation} ongelma, jossa
negatiiviset syklit ovat sallittuja.

\section*{Shortest-Augmenting-Path-algoritmi}
Shortest-Augmenting-Path-algoritmi (lyhyemmin SAP-algoritmi) etsii
minimihintaisen virtauksen lähettämällä
virtausta lyhintä reittiä kaarien hintojen suhteen lähtösolmusta kohdesolmuun.
Tällaista lyhintä reittiä kutsutaan \textit{augmentoivaksi poluksi}. Algoritmi 
on muunnelma Ford-Fulkerson-algoritmista 
maksimivirtauksen etsimiseen. Intuitiota saa miettimällä miten minimihintainen virtaus
etsittäisiin
jos virtausta lähetettäisiin vain yksi yksikkö: se olisi lyhin reitti lähtösolmusta
kohdesolmuun.
\noindent
\subsection*{Jäännösverkko}
Jotta virtausta voidaan lähettää useampia yksiköitä, täytyy olla tapa korjata aiemmin lähetettyä
virtausta. Lähetetään virtausta siis jäännösverkossa. Jäännösverkko toimii kuten Ford-Fulkersonin
jäännösverkko: virtausta voidaan peruuttaa kulkemalla jäännösverkossa vastakkaiseen suuntaan
menevää kaarta pitkin. Virtausta peruuttaessa kaaren hinta on alkuperäisen kaaren hinnan vastaluku.
Tästä seuraa että negatiivisia kaaria syntyy jäännösverkkoon vaikka niitä ei olisi verkossa alunperin.\\\\
Sanotaan että $G_f$ on virtauksen $f$ jäännösverkko. Määritellään kapasiteetit
jäännösverkossa:\\
$$U_f(uv) = \begin{cases} U(uv) - f(uv) \text{ jos } uv \in E\\
f(vu) \text{ jos } vu \in E\\
0 \text{ muuten}
\end{cases}$$
Määritellään hinnat jäännösverkossa:\\
$$C_f(uv) = \begin{cases} C(uv) \text{ jos } uv \in E\\
-C(vu) \text{ jos } vu \in E\\
0 \text{ muuten}
\end{cases}$$
Sanotaan että kaari $uv$ on jäännösverkossa jos $U_f(uv) > 0$.\\\\
\noindent
Jos jäännösverkossa on polku $s \rightsquigarrow t$ jonka pienin kapasiteetti 
on $u$, niin sitä polkua pitkin voidaan lähettää maksimissaan $u$ yksikköä
virtausta ja yhden lähetetyn virtausyksikön hinnaksi tulee polun kaarien hintojen summa.
Tämä kasvattaa virtauksen määrää. Toinen sallittu tapa muuttaa virtausta
on lähettää virtausta jotain jäännösverkon sykliä pitkin, maksimissaan syklin pienimmän
kapasiteetin verran. Laskemalla voidaan tarkistaa että nämä muutokset säilyttävät
virtauksen säilyvyys- ja kapasiteettiehdot.
\subsection*{SAP-algoritmi}
\noindent
SAP-algoritmi lähettää virtausta jäännösverkon lyhintä $s \rightsquigarrow t$-polkua
pitkin kunnes virtausta on lähetetty $k$ yksikköä tai maksimivirtaus on löydytty.
Tässä lyhin polku tarkoittaa lyhintä polkua jäännösverkon hintojen suhteen.
Lähettämällä virtausta aina lyhintä polkua pitkin jäännösverkkoon ei koskaan synny
negatiivisia syklejä. (Negatiivinen sykli on sykli jonka kaarien hintojen summa on negatiivinen.) 
Tästä seuraa myös tapa nähdä että virtaus on optimaalinen:
ainoa tapa muuttaa virtauksen hintaa muuttamatta virtaukseen määrää on lähettää virtausta
jotain sykliä pitkin. Jos negatiivisia syklejä ei ole, virtauksen hintaa ei voi muuttaa
mitenkään pienemmäksi.
Seuraavaksi todistetaan nämä väitteet.
\noindent
\subsection*{Virtauksen optimaalisuus}
\textbf{Lemma:} $k$-virtaus on optimaalinen jos jäännösverkossa ei ole sykliä jonka
hinta on negatiivinen.\\\\
\noindent
\textbf{Todistus:} Olkoon $f$ joku $k$-virtaus niin että $G_f$:ssä ei ole negatiivista
sykliä. Oletetaan että on olemassa virtaus $f'$, joka on $k$-virtaus ja jonka hinta on pienempi 
kuin $f$:n hinta. Määritellään
$(f' - f)(uv) = f'(uv) - f(uv)$. Nyt $f' - f$ on virtaus joka toteuttaa säilyvyysehdon kaikissa 
solmuissa, mukaan lukien lähtö- ja kohdesolmut, koska $f'$ ja $f$ ovat molemmat $k$-virtauksia.
Virtauksen $f' - f$ hinta on negatiivinen. Lisäksi $f' - f$ on sallittu
virtaus $G_f$:ssä. Säilyvyysehdon toteuttavan virtauksen voi hajottaa joukoksi syklejä
jotka toteuttavat säilyvyysehdon, koska jos otetaan mikä tahansa
sykli virtauksesta pois, jäljelle jää virtaus joka toteuttaa säilyvyysehdon.
Siispä kun $f' - f$ hajotetaan joukoksi syklejä, niin ainakin yhden syklin hinta on negatiivinen.
Sitä sykliä pitkin voidaan lähettää virtausta $G_f$:ssä, joten $G_f$:ssä on negatiivinen sykli.
Seuraa ristiriita, joten $f$ on optimaalinen.
\noindent
\subsubsection*{Potentiaalifunktio}
Sanotaan että potentiaalifunktio jäännösverkossa $G_f$ on funktio 
$p: V \rightarrow \mathbb{R}$, joka toteuttaa
$p(u) + C_f(uv) \ge p(v)$ kaikilla jäännösverkon kaarilla $uv$. Jos $G_f$:ssä ei ole 
negatiivista sykliä, potentiaalifunktio
on olemassa, sillä potentiaalifunktioksi voidaan ottaa $p(u)$ on lyhimmän reitin pituus
solmusta $s$ solmuun $u$.
Määritellään että kaaren $uv$ \textit{vähennetty hinta} on $C_{fr}(uv) = p(u) + C_f(uv) - p(v)$.
Nähdään että syklin hinta vähennetyissä hinnoissa on sama kuin syklin hinta koska potentiaalit
kumoavat toisensa. Nähdään myös
että kaikki vähennetyt hinnat ovat epänegatiivisia, josta on helppo nähdä että jos on
olemassa potentiaalifunktio, niin negatiivisia syklejä ei ole.\\\\
\noindent
\textbf{Lemma:} Jos jäännösverkossa ei ole negatiivista sykliä, niin lähettämällä virtausta
lyhintä $s \rightsquigarrow t$ polkua pitkin sinne ei synny negatiivista sykliä.\\\\
\noindent
\textbf{Todistus:} Kun lähetetään virtausta lyhintä $s \rightsquigarrow t$ polkua pitkin, voidaan
olemassaolevasta potentiaalifunktiosta $p$ 
päivittää potentiaalifunktio $p'$ joka on pätevä uudessa jäännösverkossa.
Olkoon $sp(u)$ lyhimmän reitin pituus
solmusta $s$ solmuun $u$ kun tarkastellaan reitin hintaa vähennetyissä hinnoissa.
Jos ei ole polkua solmusta $s$ solmuun $u$, niin $sp(u)$ on joku riittävän suuri vakio.
Uusi potentiaalifunktio on $p'(u) = p(u) + sp(u)$. Tämä toteuttaa potentiaalifunktion
ehdot kaikilla kaarilla jotka olivat jo jäännösverkossa koska $sp(u) + C_{fr}(uv) \ge sp(v)$
koska $sp$ on lyhin-reitti-funktio joten $sp(u) + p(u) + C_f(uv) - p(v) \ge sp(v)$
joten $sp(u) + p(u) + C_f(uv) \ge sp(v) + p(v)$ joten $p'(u) + C_f(uv) \ge p'(v)$.
(Huomaa että tämä pätee myös jos solmusta $s$ ei ole reittiä solmuun $u$ koska vähennetyt
hinnat ovat epänegatiivisia.)
Lisäksi verkkoon saattoi ilmestyä uusia kaaria kun lyhintä $s \rightsquigarrow t$ polkua
päinvastaiseen suuntaan menevien kaarien kapasiteetti kasvoi. Kun $u$ ja $v$ ovat
lyhimmällä polulla peräkkäin olevat solmut, $sp(u) + C_{fr}(uv) = sp(v)$ joten
$p'(u) + C_f(uv) = p'(v)$ joten $p'(v) - C_f(uv) = p'(u)$ eli potentiaalifunktio
toimii kaikilla uusilla kaarilla jotka ilmestyivät jäännösverkkoon.
\subsection*{Aikavaativuus}
Tässä määritelty SAP-algoritmi ei anna mitään tietoa tarvittavien augmentoivien
polkujen määrästä. Nähdään kuitenkin että virtauksen määrä kasvaa ainakin yhdellä, joten
augmentoivia polkuja on maksimissaan $k$. Sanotaan että verkossa on $n$ solmua,
$m$ kaarta ja suurin kaaren kapasiteetti on $U$. Periaatteessa virtausta voi lähettää 
korkeintaan $Um$ yksikköä. Lyhimmän reitin
etsiminen ei ole suoraviivaista koska verkossa on negatiivisia kaaria. Lyhimmän reitin
verkossa jossa on negatiivisia kaaria
voi etsiä Bellman-Ford-algoritmilla, jonka aikavaativuus on $O(nm)$.
Kokonaisaikavaativuudeksi tulee $O(Um^2n)$. Käytännössä käytetään
SPFA-algoritmia, joka on käytännössä hyvin nopeasti toimiva versio Bellman-Fordista.
\subsection*{Dijkstran algoritmin käyttö}
Dijkstran algoritmia ei voi käyttää ongelmaan suoraan koska verkossa on kaaria joiden hinta on
negatiivinen. Kuitenkin aiemmin esitellyn potentiaalifunktion
avulla voidaan muuttaa kaikkien kaarien hinnat epänegatiivisiksi. 
Vaikka jäännösverkko muuttuu, potentiaalifunktion voi päivittää helposti päteväksi
käyttämällä lyhimpiä polkuja jotka kuitenkin lasketaan kun lähetetään virtausta.
Jos jäännösverkossa olevan $s \rightsquigarrow t$-polun pituus on $L$, niin
sen pituus vähennetyissä hinnoissa on $L + p(s) - p(t)$ koska kaikki muut potentiaalit
kumoavat toisensa. Seuraa että lyhimmän 
$s \rightsquigarrow t$ polun voi yhtä hyvin hakea vähennettyjen hintojen verkossa.\\
Saadaan nopeampi algoritmi: Etsitään ensin joku potentiaalifunktio Bellman-Ford-algoritmilla.
Sen jälkeen käytetään Dijkstran algoritmia vähennettyjen hintojen verkossa augmentoivien polkujen
löytämiseen ja potentiaalifunktion päivittämiseen. Algoritmin aikavaativuus on \\$O(nm + Um^2 \log n)$.
\section*{Mincost-circulation}
Käytetään nimeä \textsc{Mincost-circulation} yleisemmästä versiosta ongelmalle jossa pitää
laskea minimihintainen virtaus. Mincost-circulation-ongelmassa sallitaan että
verkossa on syklejä joiden hinta on negatiivinen. Tämä tekee lähtö- ja kohdesolmujen
ja virtauksen määrän määrittämisestä turhaa, koska jos halutaan laskea aiemmin määritelty
Mincost-flow käyttäen Mincost-circulationia, voidaan lisätä kyseiseen verkkoon kaari 
kohdesolmusta lähtösolmuun jonka kapasiteetti on $k$ ja hinta on itseisarvoltaan
tarpeeksi suuri negatiivinen luku. Määritellään Mincost-circulation-ongelma: 
Annetaan virtausverkko $G = (V, E, U, C)$, jossa $U(e) \in \mathbb{Z_+}$ on 
kaaren $e$ kapasiteetti ja $C(e) \in \mathbb{R}$ on kaaren $e$ hinta. Tavoitteena on löytää minimihintainen virtaus
$f$ jota merkitään funktiona
$f : E \rightarrow \mathbb{Z}$ joka: \\\\ Minimoi $$\sum_{uv \in E} C(uv) f(uv)$$\\
Toteuttaa ehdot:\\ Säilyvyysehto
$$\sum_{v \in V} f(uv) = \sum_{v \in V} f(vu) \text{ kaikilla } u \in V$$
\\Kapasiteettiehto
$$0 \le f(uv) \le U(uv) \text{ kaikilla } uv \in E$$
Huomaa että optimaalinen hinta ei ole koskaan positiivinen koska $f(uv) = 0$ kaikilla
$uv$ toteuttaa ehdot aina ja sen hinta on $0$.
\section*{Kapasiteettiskaalaava algoritmi Mincost-circulation-ongelmalle}
Aiemmin esitellyt algoritmit mincost-flow-ongelmalle eivät toimi polynomisessa ajassa,
sillä niiden aikavaativuuksissa on kerroin $U$, joka voi olla eksponentiaalinen suhteessa
syötteen pituuteen. Nyt esitetään polynomisessa ajassa toimiva
algoritmi Mincost-circulation-ongelmalle, jonka aikavaativuudessa on 
kerroin $\log U$ kertoimen $U$ sijasta. Algoritmissa
tullaan käyttämään samaa määritelmää jäännösverkolle, potentiaalifunktiolle ja vähennettyjen
hintojen verkolle kuin aiemmin esitellyssä SAP-algoritmissa.
\subsection*{Skaalaus}
Skaalaavat algoritmit ratkaisevat ongelman ottamalla ensin huomioon vain syötteen
eniten merkitsevät bitit ja lisäämällä ratkaisun tarkkuutta bitti kerrallaan. Käytetään
tätä tekniikkaa Mincost-circulation-ongelmaan. Oletetaan että $U(uv) < 2^k$ kaikilla $uv \in E$.
Skaalaava algoritmi käyttää $k+1$ vaihetta jotka numeroidaan $0 \ldots k$. Vaiheessa 
$i$ verkko on muuten sama kuin 
alkuperäinen verkko, mutta kapasiteetit ovat $U_i(uv) = \lfloor \frac{U(uv)}{2^{k-i}} \rfloor$.
Jokaisessa vaiheessa ratkaistaan ongelma optimaalisesti sen vaiheen kapasiteeteille joten vaiheen
$k$ ratkaisu on ongleman ratkaisu.
Toisin sanoen vaiheessa $i$ ratkaistaan ongelma niin että katsotaan kapasiteettien
binääriesityksien $i$:tä eniten merkitsevää bittiä. Vaiheessa $0$ kaikki kapasiteetit ovat $0$ joten
ratkaisu on triviaali. Ongelmaksi jää siirtyminen vaiheesta $i$ vaiheeseen $i+1$. Sanotaan
että siirtyminen tehdään kahdessa vaiheessa, skaalausvaiheessa ja augmentointivaiheessa.
\subsection*{Skaalausvaihe}
Huomataan että $U_{i+1}(uv) = U_i(uv)*2$ tai $U_{i+1}(uv) = U_i(uv)*2+1$. Voidaan siis ensin kertoa
kaikkien kaarien kapasiteetit ja niissä kulkeva virtaus kahdella. Virtaus toteuttaa edelleen
kaikki ehdot ja on optimaalinen.
\subsection*{Augmentointivaihe}
Skaalausvaiheen jälkeen oltaisiin valmiita jos $U_{i+1}(uv) = U_i(uv)*2$ toteutuisi kaikilla
kaarilla. Kuitenkin kaarilla joilla $U_{i+1}(uv) = U_i(uv)*2+1$ pitää lisätä kapasiteettia 
vielä yhdellä.
Kun kasvatetaan tällaisten kaarien kapasiteettia yhdellä yksi kaari kerrallaan 
ongelma palautuu ongelmaan:
Annetaan minimihintainen virtaus verkossa. Löydä minimihintainen virtaus kun yhden kaaren 
kapasiteettia kasvatetaan yhdellä.
\subsection*{Augmentointi}
Muistetaan että virtaus on optimaalinen jos sen jäännösverkossa ei ole negatiivista
sykliä. Tästä seuraa että kaaren kapasiteetin kasvattamisen jälkeen virtaus on edelleen optimaalinen
jos se ei luo uutta negatiivista sykliä jäännösverkkoon. Toisaalta jos kapasiteetin
kasvattaminen luo negatiivisen syklin jäännösverkkoon niin sitä sykliä pitkin voidaan
augmentoida virtausta ja saadaan pienempi hinta. Siis kapasiteetin kasvattaminen 
muuttaa optimaalista virtausta
jos ja vain jos se luo negatiivisen syklin jäännösverkkoon. \\
\noindent
Otetaan jälleen avuksi potentiaalifunktio. Pidetään pätevää potentiaalifunktiota 
yllä koko algoritmin ajan. Aluksi potentiaalifunktio voi olla $p(u) = 0$ kaikilla $u$.
Skaalausvaiheessa potentiaalifunktiota ei tarvitse muuttaa koska skaalaus ei luo
uusia kaaria jäännösverkkoon eivätkä kaarien hinnat muutu.

\subsubsection*{Tapaukset}
Tarkastellaan tapauksia joita voi käydä kun kaaren $uv$ kapasitettia kasvatetaan 
yhdellä.
\subsubsection*{1. Kaari toteuttaa vanhan potentiaalifunktion}
Jos kaari toteuttaa potentiaalifunktion niin negatiivisia syklejä ei ole
eikä potentiaalifunktiota tarvitse päivittää. Huomaa että tämä tapaus tapahtuu
aina jos kaari $uv$ on jo jäännösverkossa eli $U_f(uv) > 0$ ennen kapasiteetin
lisäämistä.
\subsubsection*{2. Kaari rikkoo vanhaa potentiaalifunktiota}
Tässä tapauksessa potentiaalifunktiota
täytyy päivittää ja negatiivisia syklejä voi syntyä. 
Ennen kuin kasvatetaan kaaren $uv$ kapasiteettia
lasketaan lyhin reitti solmusta $v$ kaikkiin solmuihin vähennettyjen hintojen verkossa.
Sanotaan että lyhimmän reitin pituus solmusta $v$ solmuun $w$ vähennettyjen hintojen verkossa
on $sp(w)$. Jos solmusta $v$ ei ole reittiä solmuun $w$ niin $sp(w)$ on joku riittävän
suuri vakio.
\subsubsection*{2.1. Negatiivista sykliä ei syntynyt}
Jos syntyy negatiivinen sykli, niin se kulkee ensin uutta kaarta $uv$ pitkin ja sitten
lyhintä reittiä solmusta $v$ solmuun $u$. Tarkistamalla lyhimmän reitin pituuden solmusta
$v$ solmuun $u$ voidaan selvittää syntyikö negatiivista sykliä. Toisin sanoen negatiivista
sykliä ei syntynyt jos $sp(u) + p(u) + C(uv) - p(v) \ge 0$. Potentiaalifunktio
pitää kuitenkin korjata vaikka negatiivista sykliä ei syntynyt. Asetetaan uudeksi
potentiaalifunktioksi $p'(w) = p(w) + sp(w)$. Tämä toteuttaa
potentiaalifunktion ehdot kaikilla jäännösverkossa jo olevilla kaarilla $wz$ koska 
$sp(w) + C_{fr}(wz) \ge sp(z)$ koska $sp$ on lyhin-reitti-funktio
joten $p(w) + sp(w) + C_f(wz) \ge p(z) + sp(z)$ joten $p'(w) + C_f(wz) \ge p'(z)$.
(Huomaa että tämä pätee myös jos solmusta $v$ ei ole reittiä solmuun $w$ koska vähennetyt
hinnat ovat epänegatiivisia.) Uusi potentiaalifunktio toteuttaa potentiaalifunktion
ehdot myös uudella kaarella $uv$ koska $sp(u) + p(u) + C(uv) - p(v) \ge 0$ joten
$p(u) + sp(u) + C_f(uv) \ge p(v) + sp(v)$ koska $sp(v) = 0$ ja $C_f(uv) = C(uv)$
joten $p'(u) + C_f(uv) \ge p'(v)$.
\subsubsection*{2.2. Negatiivinen sykli syntyi}
Negatiivinen sykli syntyy jos $sp(u) + p(u) + C(uv) - p(v) < 0$. Tässä tapauksessa 
virtausta lähetetään yksi yksikkö sykliä joka koostuu kaaresta $uv$ ja lyhimmästä
$v \rightsquigarrow u$-polusta pitkin. Kun virtausta on lähetetty tätä sykliä
pitkin, verkossa ei ole enää negatiivisia syklejä. Todistetaan se näyttämällä pätevä
potentiaalifunktio. Jälleen kerran uusi potentiaalifunktio on $p'(w) = p(w) + sp(w)$.
Edellisen kohdan analyysin perusteella tämä potentiaalifunktio on pätevä kaikille 
jäännösverkossa jo olleille kaarille. Jäännösverkkoon kuitenkin saattoi ilmestyä
uusia kaaria jotka menevät vastakkaiseen suuntaan lyhintä $v \rightsquigarrow u$-polkua.
Otetaan lyhimmällä $v \rightsquigarrow u$-polulla peräkkäin olevat solmut $w$ ja $z$.
Nyt $sp(z) = sp(w) + C_{fr}(wz)$ joten $sp(z) = sp(w) + p(w) + C_f(wz) - p(z)$ joten
$p(z) + sp(z) = p(w) + sp(w) + C_f(wz)$ joten $p'(z) = p'(w) + C_f(wz)$ joten 
$p'(z) + C_f(zw) = p'(w)$ eli potentiaalifunktio pätee kaarilla jotka menevät 
vastakkaiseen suuntaan lyhintä $v \rightsquigarrow u$-polkua. Lisäksi jäännösverkkoon
tulee kaari $vu$. $sp(u) + p(u) + C(uv) - p(v) < 0$ joten 
$sp(u) + p(u) + C(uv) < p(v)$ joten $p(v) + C(vu) > p(u) + sp(u)$ joten
$p'(v) + C(vu) > p'(u)$ koska $sp(v) = 0$. Huomaa että jäännösverkkoon ei tule kaarta
$uv$.
\\\\
\noindent
Nyt kaikkiin yhden kaaren augmentoinnissa tapahtuviin tapauksiin on esitetty ja analysoitu
algoritmi joten augmentointivaihe on valmis. Täten koko algoritmi on valmis, koska siirtymä
vaiheesta $i$ vaiheeseen $i+1$ on esitetty kokonaan ja on todettu että vaiheen 0 ratkaisu
on triviaali.
\subsection*{Aikavaativuus}
Algoritmissa on $\log_2 U$ vaihetta. Jokaisessa vaiheessa augmentointi dominoi aikavaativuutta.
Jokaisessa vaiheessa augmentoidaan korkeintaan
$m$ kaarta. Kaaren augmentoinnissa haetaan lyhimmät reitit vähennettyjen hintojen
verkossa jossa siihen voi käyttää Dijkstran algoritmia, eli lyhimpien reittien haun
aikavaativuudeksi tulee $O(m \log n)$. Yhteensä aikavaativuudeksi tulee $O(\log U m^2 \log n)$, joka
on polynominen aikavaativuus ja teoriassa parempi kuin aiemmin käsitellyt algoritmit mincost-flow-ongelmaan.
Vertailun vuoksi maksimivirtauksen laskemiseen käytetyn Edmonds-Karp algoritmin aikavaativuus
on $O(m^2n)$.
\subsection*{Lisähuomio potentiaaleista}
Monessa kohdassa solmujen potentiaaleihin lisätään jotain epämääräisiä lukuja joiden suuruutta
algoritmin aikana ei ole analysoitu. Potentiaalien kasvaminen räjähdysmäisesti osoittautuu
myös käytännön ongelmaksi. Potentiaalifunktion kuitenkin pystyy korjaamaan $O(m \log n)$ ajassa
sellaiseksi että kaikki potentiaalit ovat rajoitettuja suhteessa syötteen kokoon.
Korjaaminen tapahtuu seuraavalla tavalla: Otetaan joku solmu $v$ jonka potentiaalia 
ei ole vielä korjattu. Lasketaan siitä
Dijkstran ja potentiaalifunktion avulla lyhimmät polut solmuihin joihin pääsee siitä ja joita ei ole
vielä käsitelty. Asetetaan näiden solmujen potentiaaliksi lyhin reitti solmusta $v$ + 
suurin korjattu potentiaali ennen tätä vaihetta. Nyt saadaan pätevä potentiaalifunktio, jonka
itseisarvo on kaikissa solmuissa korkeintaan $nC$, jossa $C$ on suurin hintojen itseisarvo. Korjaaminen
voidaan tehdä vaikka jokaisen potentiaalien päivityksen jälkeen eikä se vaikuta aikavaativuuteen koska
Dijkstra ajetaan kuitenkin aina kun potentiaaleja päivitetään.
\end{document}




